% [Overleaf] https://www.overleaf.com/read/pjymmthhrbyc
% [YouTube] https://youtu.be/z_I-ht9u_L8
% [GitHub] https://github.com/nobucshirai/infona2020_slide_13
\documentclass[dvipdfmx,aspectratio=169,20pt]{beamer}
\usepackage{bxdpx-beamer}

% Beamer theme
\usetheme{Boadilla}

%%%%% JAPANESE FONT SETTINGS %%%%%
\usepackage[utf8]{inputenc}
\usepackage{pxjahyper}
\renewcommand{\kanjifamilydefault}{\gtdefault} % for Gothic Japanese fonts
\newcommand{\myfontsetting}[3]{{\fontsize{#1}{#2}\selectfont #3}}
\usepackage[deluxe,uplatex]{otf} % needed to use super bold Japanese fonts
\usepackage[unicode,noto-otc]{pxchfon} % needed to use super bold Japanese fonts
%%%%%%%%%%%%%%%%%%%%%%%%%%%%%%%%%%

%%%%% SETTINGS FOR MATH SYMBOLS %%%%%
\usepackage{amsmath,amssymb}
\usepackage{bm}
%\usepackage{graphicx}
\usepackage{latexsym}
\usefonttheme{professionalfonts} % use Serif fonts for equations
%%%%%%%%%%%%%%%%%%%%%%%%%%%%%%%%%%%%%

\usepackage{fancybox,ascmac}
\usepackage{url}
\usepackage[many]{tcolorbox}

%%%%% ALGORITHM SETTING %%%%%
\usepackage{algorithm}
\usepackage[noend]{algorithmic}
\algsetup{linenosize=\color{fg!50}\fontsize{8pt}{8pt}\selectfont}
\renewcommand\algorithmicdo{\bfseries :}
\renewcommand\algorithmicthen{\bfseries :}
\renewcommand\algorithmicrequire{\textbf{Input:}}
\renewcommand\algorithmicensure{\textbf{Output:}}
\renewcommand{\algorithmicprint}{\textbf{break}}
%%%%%%%%%%%%%%%%%%%%%%%%%%%%%
\definecolor{myblue1}{RGB}{45,130,200}
\definecolor{myblue2}{RGB}{26,89,142}
\setbeamertemplate{navigation symbols}{}
\setbeamercolor*{structure}{fg=myblue1,bg=blue}
\setbeamercolor{block title}{fg=myblue1!50!black}
\setbeamercolor*{block title example}{fg=white,bg=myblue2}
\setbeamercolor*{palette primary}{use=structure,fg=white,bg=structure.fg}
\setbeamercolor*{palette secondary}{use=structure,fg=white,bg=structure.fg!75!black}
\setbeamercolor*{palette tertiary}{use=structure,fg=white,bg=structure.fg!50!black}
\setbeamercolor*{palette quaternary}{fg=black,bg=myblue1}

\setbeamerfont{alerted text}{series=\bfseries}
\setbeamerfont{section in toc}{series=\mdseries}
\setbeamerfont{frametitle}{size=\Large,series=\bfseries}
\setbeamerfont{title}{size=\LARGE,series=\bfseries}
\setbeamerfont{date}{size=\small}

\setbeamertemplate{block title}[shadow=false]
\setbeamertemplate{blocks}[rounded][shadow=false]

%%%%% BEAMER FOOTLINE SETTINGS %%%%%%
\setbeamertemplate{footline}[frame number]{}
\setbeamerfont{footline}{size=\bf\footnotesize\small}
%%%%%%%%%%%%%%%%%%%%%%%%%%%%%%%%%%%%%

%%%%% BEAMER ITEM SETTINGS %%%%%
\setbeamertemplate{itemize item}[circle]
\setbeamertemplate{itemize subitem}[triangle]
\setbeamertemplate{enumerate item}[circle]
%%%%%%%%%%%%%%%%%%%%%%%%%%%%%%%%

\begin{document}

%%%%%%%%%%%%%%%%%%%%%%%%%%%%%%%%
\begin{frame}
%%%%% START_TAG B %%%%%
%\noindent{\bf X\hspace{-.1em}I\hspace{-.1em}I-B.}
\frametitle{[問題] X\hspace{-.1em}I\hspace{-.1em}I-B}

\myfontsetting{15pt}{18pt}{
1階の常微分方程式 $\frac{dy}{dt}=y$ を4次のルンゲ・クッタ法を用いて数値的に解く場合を考える。初期値を $(t_0,y_0)=(0,1)$、時間の刻み幅を $h=0.01$ とした時、 $y(10)$ の数値解を求めるプログラムを作成し解答せよ。また $y(10)$ の真値である $22026.4657948067$ と比べることにより相対誤差を求めよ。数値解は有効数字10桁で11桁目を四捨五入し、相対誤差は有効数字3桁で4桁目を四捨五入して答えよ。
}\\
\myfontsetting{12pt}{12pt}{
作成したプログラムも提出すること。プログラミング言語は問わない。
}
%%%%% END_TAG B %%%%%
\end{frame}
%%%%%%%%%%%%%%%%%%%%%%%%%%%%%%%%
\begin{frame}
\frametitle{[略解] X\hspace{-.1em}I\hspace{-.1em}I-B}
%\vspace{-0.5cm}

$\tilde{y}(10)=22026.46578$

\vspace{0.5cm}

$\frac{|\tilde{y}(10)-y(10)|}{|y(10)|} = 8.26\times 10^{-10}$

\end{frame}
%%%%%%%%%%%%%%%%%%%%%%%%%%%%%%%%
\begin{frame}
\frametitle{{\large [手法] 4次のルンゲ・クッタ法}}
    \begin{block}{{\bf\small 4次のルンゲ・クッタ法}
    \myfontsetting{13pt}{18pt}{(Fourth-order Runge-Kutta method)}}
        \myfontsetting{12pt}{15pt}{
        \begin{algorithmic}[1]
            \REQUIRE $h, t_0, y_0, k$, $f(t,y)$ \hspace{2mm} \myfontsetting{10pt}{10pt}{\bf [$f(t,y)$ は関数]}
            \ENSURE $y_1,y_2,\dots, y_k$ \hspace{2mm} \myfontsetting{10pt}{10pt}{\bf [$t=t_0 + kh$ での $y$ の値の近似値]} \FOR{$i=0,1,\dots,k-1$}
            \STATE $v_1 \leftarrow f(t_i,y_i)h$
            \STATE $v_2 \leftarrow f(t_i + \frac{h}{2},y_i+\frac{v_1}{2})h$
            \STATE $v_3 \leftarrow f(t_n + \frac{h}{2}, y_i + \frac{v_2}{2})h$
            \STATE $v_4 \leftarrow f(t_i + h, y_i + v_3)h$
            \STATE $y_{i+1} \leftarrow y_i + \frac{1}{6}\bigl(v_1 + 2 (v_2 + v_3) + v_4\bigr)$
            \STATE $t_{i+1} \leftarrow t_{i} + h$
            \ENDFOR
        \end{algorithmic}
        }
    \end{block}
\end{frame}
%%%%%%%%%%%%%%%%%%%%%%%%%%%%%%%%
%タイトルページ

\title{\myfontsetting{32pt}{32pt}{常微分方程式の数値解法 (2)}}

\titlegraphic{\vspace{-7mm}\flushright\includegraphics[width=1.8cm,height=1.8cm]{hattari_kun_good_org.eps}}

\setbeamertemplate{title page}{%
    \begin{flushright}
        \usebeamercolor[fg]{titlegraphic}\inserttitlegraphic
    \end{flushright}
    \vspace{-0.6cm}
    \hspace{1.5cm}{\selectfont\usebeamerfont{subtitle} \usebeamercolor[fg]{subtitle} [\href{https://youtu.be/z_I-ht9u_L8}{数値解析 第13回}] \par}
    \vspace{0.5cm}
    %\vspace{2.5em}
    {\centering\usebeamerfont{title} \usebeamercolor[fg]{title} \inserttitle \par}
    \vspace{0.5cm}
    \begin{center}
        \myfontsetting{18pt}{18pt}{テイラー展開から旨い数値スキームを構成する}
    \end{center}
}

\date[\todey]{}

\frame{\titlepage}

%%%%%%%%%%%%%%%%%%%%%%%%%%%%%%%%
\begin{frame}
\frametitle{\myfontsetting{22pt}{22pt}{初期値問題に対する数値スキームの構成法}}
\begin{block}{\myfontsetting{20pt}{20pt}{\bf テイラー展開を用いた数値スキームの構成}}
%\item \myfontsetting{12pt}{12pt}{{\bf [形式1]} \hspace{3mm} $\displaystyle y(t)=\sum_{k=0}^\infty \frac{1}{k!}\left.\frac{d^k y(x)}{dt^k}\right|_{x=a} (x-a)^k$}
%\item 
\myfontsetting{14pt}{14pt}{テイラー展開 \myfontsetting{10pt}{10pt}{$\displaystyle y(t+\varDelta t)=\sum_{k=0}^\infty \frac{1}{k!}\frac{d^k y(t)}{dt^k}\varDelta t^k$} に対して小さくとった刻み幅 $\varDelta t$ を準備し、 $\varDelta t$ の高次の項を無視する近似を用いて{\bf 数値スキーム} \myfontsetting{10pt}{10pt}{(Numerical scheme)} を構成する。$\varDelta t$ の $k$ 乗まで考慮した数値スキームを{\bf k次の数値スキーム}と呼ぶ。}
\end{block}
\begin{itemize}
    \item \myfontsetting{12pt}{12pt}{$\varDelta t$ の1次まで考慮した \myfontsetting{10pt}{10pt}{$y(t+\varDelta t)\sim y(t)+\frac{dy}{dt}\varDelta t$} から微分方程式 \myfontsetting{10pt}{10pt}{$\frac{dy}{dt}=f(t,y)$} に対する{\bf オイラー法} \myfontsetting{10pt}{10pt}{$y(t+\varDelta t)\sim y(t)+f(t,y)\varDelta t$} \myfontsetting{10pt}{10pt}{(1次の数値スキーム)} が得られる}
    \begin{itemize}
        \item \myfontsetting{12pt}{12pt}{$\varDelta t$ の2次以上の項も考慮するとさらに旨い近似式が得られる}
    \end{itemize}
\end{itemize}

\end{frame}
%%%%%%%%%%%%%%%%%%%%%%%%%%%%%%%%
\begin{frame}
\frametitle{\large 高次の数値スキームを得るための方法}
\begin{itemize}
    \item \myfontsetting{14pt}{14pt}{数値スキームを構成する計算過程で\myfontsetting{12pt}{12pt}{$\displaystyle \frac{dy(t)}{dt}$}が現れた場合は $f(t,y)$ に置き換える}
    \item \myfontsetting{14pt}{14pt}{テイラー展開において $\varDelta t$ の2次以上の項を考慮するには \myfontsetting{12pt}{12pt}{$\displaystyle \frac{d^k y(t)}{dt^k}$ $(k\ge 2)$} を計算する必要がある}
    \item \myfontsetting{14pt}{14pt}{$k=2$ の時}
    \begin{itemize}
        \item \myfontsetting{12pt}{12pt}{$\displaystyle\frac{d^2y(t)}{dt^2}=\frac{d}{dt}\left(\frac{dy}{dt} \right) = \frac{d}{dt}f(t,y)$} 
    \end{itemize}
    \item \myfontsetting{14pt}{14pt}{一般の $k$ について}
    \begin{itemize}
        \item \myfontsetting{12pt}{12pt}{$\displaystyle\frac{d^ky(t)}{dt^k}=\frac{d^{k-1}}{dt^{k-1}}f(t,y)$} 
    \end{itemize}
\end{itemize}
\end{frame}
%%%%%%%%%%%%%%%%%%%%%%%%%%%%%%%%
\begin{frame}
\frametitle{[問題] X\hspace{-.1em}I\hspace{-.1em}I\hspace{-.1em}I-A}

%%%%% START_TAG A %%%%%
%\noindent{\bf [X\hspace{-.1em}I\hspace{-.1em}I\hspace{-.1em}I. 常微分方程式の数値解法 (2)]}%RETURN

%\noindent{\bf X\hspace{-.1em}I\hspace{-.1em}I\hspace{-.1em}I-A.}
常微分方程式 \myfontsetting{15pt}{15pt}{$\displaystyle \frac{dy}{dt} = f(t,y)$} に対する2次のルンゲ・クッタ法が2次の数値スキームであることを示せ。

%%%%% END_TAG A %%%%%
\end{frame}
%%%%%%%%%%%%%%%%%%%%%%%%%%%%%%%%
\begin{frame}
\frametitle{[略解] X\hspace{-.1em}I\hspace{-.1em}I\hspace{-.1em}I-A}
%\vspace{-0.5cm}

\myfontsetting{18pt}{18pt}{2次のルンゲ・クッタ法の数値スキームを $\varDelta t << t$ としてテイラー展開すると

\vspace{5mm}

\myfontsetting{15pt}{15pt}{$\tilde{y}(t+\varDelta t) \sim y(t) + f\varDelta t + \frac{1}{2}\left\{ f_t + f_y f \right\}\varDelta t^2 + \frac{1}{4} \left\{ f_{tt} + f_{ty} f + f_{yy} f^2 \right\} \varDelta t^3$}

\vspace{5mm}

となる。これは $y(t+\varDelta t)$ のテイラー展開の $\varDelta t^k$ $k \ge 0$ と係数を比較すると $\varDelta t^2$ の項の係数まで一致しているため2次の数値スキームである。}

\end{frame}
%%%%%%%%%%%%%%%%%%%%%%%%%%%%%%%%
\begin{frame}
\frametitle{\large \myfontsetting{20pt}{20pt}{$ \frac{df(t,y)}{dt}$}の計算方法 (1)---多変数関数の微分}
\begin{block}{\myfontsetting{20pt}{20pt}{\bf 偏微分}{\small (Partial derivative)}}
\myfontsetting{14pt}{16pt}{
2つ以上の変数を含む関数を考える。選んだ1つの変数の微分を考える時に選んだ変数以外の変数の値を固定して定義した微分は通常の微分 (常微分) と区別して{\bf 偏微分}と呼ぶ。
}
\end{block}
\begin{itemize}
    %\setlength{\itemsep}{0.5cm}
    \item \myfontsetting{14pt}{16pt}{$f(t,y)$ の $t$ 偏微分は \myfontsetting{10pt}{10pt}{$\displaystyle \frac{\partial f(t,y)}{\partial t} = \lim_{\varDelta t\to 0} \frac{f(t+\varDelta t,y) - f(t,y)}{\varDelta t}$} のように定義する \myfontsetting{10pt}{10pt}{(簡略化のため $t$ による偏微分は $f_t(t,y)$ とも表記する)}}
    \begin{itemize}
        \item \myfontsetting{12pt}{14pt}{$y$ が $t$ の関数であり $y(t)$ と表される場合でも $f(t,y)$ の $t$ 偏微分を考える場合は $y$ は独立変数のように扱い定数として値を固定する}
    \end{itemize}
\end{itemize}
\end{frame}
%%%%%%%%%%%%%%%%%%%%%%%%%%%%%%%%
\begin{frame}
\frametitle{\large \myfontsetting{22pt}{22pt}{$ \frac{df(t,y)}{dt}$} の計算方法 (2)---連鎖律}
\begin{block}{\myfontsetting{20pt}{20pt}{\bf 連鎖律}{\small (Chain rule)}}
\myfontsetting{14pt}{16pt}{
合成関数の微分に関する微分規則であり、ある変数の常微分または偏微分を合成関数の微分の積の和で表すことができる。例えば変数 $t$ の関数 $x(t)$, $y(t)$ を含む関数 $f(x,y)$ を考えた時、連鎖律は
\myfontsetting{10pt}{10pt}{$\displaystyle \frac{df(x,y)}{dt} = \frac{\partial f(x,y)}{\partial x}\frac{dx}{dt} + \frac{\partial f(x,y)}{\partial y}\frac{dy}{dt} = f_x(x,y)\frac{dx}{dt} + f_y(x,y) \frac{dy}{dt}$} となる。
}
\end{block}
\begin{itemize}
    \item \myfontsetting{14pt}{16pt}{上記の例において $x(t) = t$ とおくと \myfontsetting{10pt}{10pt}{$\displaystyle \frac{df(t,y)}{dt} = f_t(t,y) + f_y(t,y) \frac{dy}{dt}$ $= f_t(t,y) + f_y(t,y) f(t,y)$} が成り立つ} 
\end{itemize}

\end{frame}
%%%%%%%%%%%%%%%%%%%%%%%%%%%%%%%%
\begin{frame}
\frametitle{{\large [手法] 2次のルンゲ・クッタ法}}
    \begin{block}{{\bf\small 2次のルンゲ・クッタ法}
    \myfontsetting{13pt}{18pt}{(Second-order Runge-Kutta method)}}
        \myfontsetting{15pt}{18pt}{
        \begin{algorithmic}[1]
            \REQUIRE $h, t_0, y_0, k$, $f(t,y)$ \hspace{2mm} \myfontsetting{10pt}{10pt}{\bf [$f(t,y)$ は関数]}
            \ENSURE $y_1,y_2,\dots, y_k$ \hspace{2mm} \myfontsetting{10pt}{10pt}{\bf [$t=t_0 + kh$ での $y$ の値の近似値]} \FOR{$i=0,1,\dots,k-1$}
            \STATE $v_1 \leftarrow f(t_i,y_i)h$
            \STATE $v_2 \leftarrow f(t_i + h, y_i+v_1)h$
            \STATE $y_{i+1} \leftarrow y_i + \frac{1}{2}\bigl(v_1 + v_2 \bigr)$
            \STATE $t_{i+1} \leftarrow t_{i} + h$
            \ENDFOR
        \end{algorithmic}
        }
    \end{block}
\end{frame}
%%%%%%%%%%%%%%%%%%%%%%%%%%%%%%%%
\begin{frame}
%%%%% PASTE_START_TAG B %%%%%
%\noindent{\bf X\hspace{-.1em}I\hspace{-.1em}I\hspace{-.1em}I-B.}
\frametitle{[問題] X\hspace{-.1em}I\hspace{-.1em}I\hspace{-.1em}I-B (1)}
\myfontsetting{18pt}{18pt}{
単振動を表す運動方程式 $m\frac{d^2 x(t)}{dt^2}= - k x(t)$ を考える。
$x(t)$ の一般解を求めよ。また速度 $v(t)=\frac{dx}{dt}$ を新たに定義し $v(t)$ の一般解を求めよ。
}
\end{frame}
%\\
\begin{frame}
\frametitle{[問題] X\hspace{-.1em}I\hspace{-.1em}I\hspace{-.1em}I-B (2)}
\myfontsetting{18pt}{18pt}{
以下の問題では質量およびバネ定数の値を $m=1$, $k=1$ とおく。
問1で得られた $x(t)$ および $v(t)$ について $t=0$ で $x(0)=1$, $v(0)=0$ とした時の解を求めよ。
}
\end{frame}
%\\
\begin{frame}
\frametitle{[問題] X\hspace{-.1em}I\hspace{-.1em}I\hspace{-.1em}I-B (3)}
\myfontsetting{13pt}{15pt}{
問1で示した2階の常微分方程式を2本の1階微分方程式に書き換え、蛙飛び法を用いて数値的に解く場合を考える。
時間の刻み幅 $\varDelta t=0.01$、 初期値を $(x_0,v_0)=(1,- \sin \left(- \frac{\varDelta t}{2}\right) )$ ($x_0$ は $x(0)$ に、 $v_0$ は $v\left(- \frac{\varDelta t}{2}\right)$ に対応する)、質量およびバネ定数の値を $m=1$, $k=1$ と設定した時、蛙飛び法を用いて時間発展を計算するプログラムを作成し時刻 $t=10$ での位置と速度の値を相対誤差とともに数値で求めよ。数値は有効数字3桁で4桁目を四捨五入して答えよ。
}\\
\myfontsetting{12pt}{12pt}{
作成したプログラムも提出すること。プログラミング言語は問わない。
}
%%%%% PASTE_END_TAG B %%%%%
\end{frame}
%%%%%%%%%%%%%%%%%%%%%%%%%%%%%%%%
\begin{frame}
\frametitle{\myfontsetting{18pt}{18pt}{[手法] 蛙飛び法---\myfontsetting{15pt}{15pt}{$\frac{dv}{dt}=f(x,v)$,  $\frac{dx}{dt}=g(x,v)$} に対する解法}}
\begin{block}{{\bf\small 蛙飛び法}
\myfontsetting{13pt}{18pt}{(Leap frog method)}}
    \myfontsetting{13pt}{15pt}{
    \begin{algorithmic}[1]
        \REQUIRE $\varDelta t, t_0, x_0, v_0, k$, $f(x,v)$, $g(x,v)$ \hspace{2mm} \myfontsetting{10pt}{10pt}{\bf [$f(x,v)$, $g(x,v)$ は関数]}
        \ENSURE \myfontsetting{10pt}{10pt}{$(x_1,v_1+f(x_1,v_1)\frac{\varDelta t}{2}), \dots, (x_k,v_k+f(x_k,v_k)\frac{\varDelta t}{2})$} \myfontsetting{8pt}{8pt}{\bf [$t=t_0 + k\varDelta t$ での $x$, $v$ の値の近似値]} \FOR{$i=0,1,\dots,k-1$}
        \STATE $v_{i+1} \leftarrow v_i + f(x_i,v_i)\varDelta t$
        \STATE $x_{i+1} \leftarrow x_i + g(x_i,v_{i+1})\varDelta t$
        \STATE $t_{i+1} \leftarrow t_{i} + \varDelta t$
        \ENDFOR
    \end{algorithmic}
    }
\end{block}
\vspace{-5mm}
\begin{itemize}
    \item \myfontsetting{10pt}{10pt}{$v(t)$ と $x(t)$ の時刻が $\frac{\varDelta t}{2}$ だけずれた状態で並走していると考える}
    \item \myfontsetting{10pt}{10pt}{$x_0$ を $x(t)$ の $t=0$ での値として与える場合 $v_0$ は $v(t)$ の $t=-\frac{\varDelta t}{2}$ での値として設定する。$v(t_k)$ を出すにはオイラー法で $v_k$ を $\frac{\varDelta t}{2}$ だけ進める ($v(t_k)\sim v_k + f(x_k,v_k)\frac{\varDelta t}{2}$)}

\end{itemize}
\end{frame}
%%%%%%%%%%%%%%%%%%%%%%%%%%%%%%%%
\end{document}
